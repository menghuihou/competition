
\documentclass{article}
\usepackage[UTF8]{ctex}
\usepackage[textwidth=444bp,vmargin=2.5cm]{geometry}%设置页边距
\usepackage{array} %主要是增加列样式选项
\usepackage[dvipsnames]{xcolor}%颜色宏包
\usepackage{graphicx}%图片宏包
\usepackage{amsmath}%公式宏包
\usepackage[T1]{fontenc}    
\usepackage{newtxtext, newtxmath}  %两种使用Times New Roman 字体的方法

\begin{document}
%----------- 中文摘要 ----------
\newpage

\begin{center}
\lunwenbiaoti

\vspace{2ex}
\zhaiyao
\end{center}

开头段:需要充分概括论文内容,一般两到三句话即可,长度控制在三至五行。

问题一中,解决了什么问题;应用了什么方法;得到了什么结果。

问题二中,解决了什么问题;应用了什么方法;得到了什么结果。

问题三中,解决了什么问题;应用了什么方法;得到了什么结果。

结尾段:可以总结下全文,也可以介绍下你的论文的亮点,也可以对类似的问题进行适当的推广。

\begin{guanjianci}
关键词一 \quad 关键词二 \quad 关键词三
\end{guanjianci}

%----------- 正文 ----------
%----------- 一、问题重述 ----------
\newpage
\section{一、问题重述}
数学建模比赛论文是要我们解决一道给定的问题,所以正文部分一般应从问题重述开始,一般确定选题后就可以开始写这一部分了。

这部分的内容是将原问题进行整理,将问题背景和题目分开陈述即可,所以基本没啥难度。

本部分的目的是要吸引读者读下去,所以文字不可冗长,内容选择不要过于分散、琐碎,措辞要精练。

注意:在写这部分的内容时,绝对不可照抄原题!(论文会查重)

应为:在仔细理解了问题的基础上,用自己的语言重新将问题描述一遍。语言需要简明扼要,没有必要像原题一样面面俱到。

\subsection{问题背景}
这是你的内容这是你的内容这是你的内容这是你的内容这是你的内容这是你的内容这是你的内容
\subsection{问题重述}
这是你的内容这是你的内容这是你的内容这是你的内容这是你的内容这是你的内容这是你的内容
%----------- 二、问题分析 ----------
\section{二、问题分析}
\subsection{问题一的分析}
从实际问题到模型建立是一种从具体到抽象的思维过程,问题分析这一部分就是沟通这一过程的桥梁,因为它反映了建模者对于问题的认识程度如何,也体现了解决问题的雏形,起着承上启下的作用,也很能反应出建模者的综合水平。

这部分的内容应包括:题目中包含的信息和条件,利用信息和条件对题目做整体分析,确定用什么方法建立模型,一般是每个问题单独分析一小节,分析过程要简明扼要, 不需要放结论。

建议在文字说明的同时用图形或图表(例如流程图)列出思维过程,这会使你的思维显得很清晰,让人觉得一目了然。

(注意:问题分析这一部分放置的位置比较灵活,可以放在问题重述后面作为单独的一节(见到的频率最高),也可以放在模型假设和符号说明后面作为单独的一节,还可以针对每个问题将其写在模型建立中。)

这是你的内容这是你的内容这是你的内容这是你的内容这是你的内容这是你的内容这是你的内容
\subsection{问题二的分析}
这是你的内容这是你的内容这是你的内容这是你的内容这是你的内容这是你的内容这是你的内容
\subsection{问题三的分析}
这是你的内容这是你的内容这是你的内容这是你的内容这是你的内容这是你的内容这是你的内容
%----------- 三、模型假设 ----------
\section{三、模型假设}
以下是6类常见的模型假设:
1.题目明确给出的假设条件
2.排除生活中的小概率事件(例如黑天鹅事件、非正常情况)
3.仅考虑问题中的核心因素,不考虑次要因素的影响
4.使用的模型中要求的假设
5.对模型中的参数形式(或者分布)进行假设
6.和题目联系很紧密的一些假设,主要是为了简化模型

这是你的内容这是你的内容这是你的内容这是你的内容这是你的内容这是你的内容这是你的内容


%----------- 四、符号说明 ----------
\section{四、符号说明}
%使用三线表格最好~
\begin{table}[h]%htbp表示的意思是latex会尽量满足排在前面的浮动格式,就是h-t-b-p这个顺序,让排版的效果尽量好。
    \centering
    \begin{tabular}{p{2.0cm}<{\centering}p{9.0cm}<{\centering}p{2.0cm}<{\centering}}
 %指定单元格宽度, 并且水平居中。
    \hline
    符号 & 说明 & 单位 \\ %换行 
    \hline
    $\int$ & 积分符号 &  \\ %把你的符号写在这
    $W_0$ & 区分高峰和低峰的一个临界值 &  \\ %把你的符号写在这
    $M_t$ &  简单移动平均项 &  \\ %把你的符号写在这
    \hline
    \end{tabular}
\end{table}
本部分是对模型中使用的重要变量进行说明,一般排版时要放到一张表格中。

注意:第一:不需要把所有变量都放到这个表里面,模型中用到的临时变量可以不放。第二:下文中首次出现这些变量时也要进行解释,不然会降低文章的可读性。

%----------- 五、模型的建立与求解 ----------
\section{五、模型的建立与求解}

(注意:这个部分里面的标题可根据你的论文内容进行调整,我这里给的是一个通用的模版)

\subsection{问题一模型的建立与求解}
\subsubsection{模型的建立}
模型建立是将原问题抽象成用数学语言的表达式,它一定是在先前的问题分析和模型假设的基础上得来的。因为比赛时间很紧,大多时候我们都是使用别人已经建立好的模型。这部分一定要将题目问的问题和模型紧密结合起来,切忌随意套用模型。我们还可以对已有模型的某一方面进行改进或者优化,或者建立不同的模型解决同一个问题,这样就是论文的创新和亮点。
\subsubsection{模型的求解}
把实际问题归结为一定的数学模型后,就要利用数学模型求解所提出的实际问题了。一般需要借助计算机软件进行求解,例如常用的软件有Matlab, Spss, Lingo, Excel, Stata, Python等。求解完成后,得到的求解结果应该规范准确并且醒目,若求解结果过长,最好编入附录里。(注意:如果使用智能优化算法或者数值计算方法求解的话,需要简要阐明算法的计算步骤)
\subsection{问题二模型的建立与求解}

\subsection{问题三模型的建立与求解}

公式\eqref{速度公式}是我们使用的速度公式。

\begin{equation}
p_{n}=\rho^{n}p_{0},\quad n=1,2,\cdot\cdot,K
\label{速度公式}
\end{equation}
%直接点击右上角“公式编辑器”,上传公式截图即可一键复制LaTeX代码

\begin{equation}
v_{in}\left( t \right) =\begin{cases}
 CA\sqrt{\frac{2\varDelta P}{\rho \left( t \right)}},A\text{处单项阀开启}\\
 0,             A\text{处单项阀关闭}\\
\end{cases}
\end{equation}

\begin{figure}[h]
\centering
\includegraphics[width=0.5\textwidth]{slagerlogo}
\caption{更多模板和论文工具尽在Slager:slager.cn}
\end{figure}

%----------- 六、模型的分析与检验 ----------
\section{六、模型的分析与检验}

模型的分析与检验的内容也可以放到模型的建立与求解部分,这里我们单独抽出来进行讲解,因为这部分往往是论文的加分项,很多优秀论文也会单独抽出一节来对这个内容进行讨论。

模型的分析 :在建模比赛中模型分析主要有两种,一个是灵敏度(性)分析,另一个是误差分析。灵敏度分析是研究与分析一个系统(或模型)的状态或输出变化对系统参数或周围条件变化的敏感程度的方法。其通用的步骤是:控制其他参数不变的情况下,改变模型中某个重要参数的值,然后观察模型的结果的变化情况。误差分析是指分析模型中的误差来源,或者估算模型中存在的误差,一般用于预测问题或者数值计算类问题。

模型的检验:模型检验可以分为两种,一种是使用模型之前应该进行的检验,例如层次分析法中一致性检验,灰色预测中的准指数规律的检验,这部分内容应该放在模型的建立部分;另一种是使用了模型后对模型的结果进行检验,数模中最常见的是稳定性检验,实际上这里的稳定性检验和前面的灵敏度分析非常类似。

\section{七、模型的评价、改进与推广}
注:本部分的标题需要根据你的内容进行调整,例如:如果你没有写模型推广的话,就直接把标题写成模型的评价与改进。很多论文也把这部分的内容直接统称为“模型评价”部分,也是可以的。

\subsection{模型的优点}
优缺点是必须要写的内容,改进和推广是可选的,但还是建议大家写,实力比较强的建模者可以在这一块充分发挥,这部分对于整个论文的作用在于画龙点睛。
\subsection{模型的缺点}
缺点写的个数要比优点少
\subsection{模型的改进}
主要是针对模型中缺点有哪些可以改进的地方\cite{risken1996fokker};
\subsection{模型的推广}
将原题的要求进行扩展\cite{rossler1979equation},进一步讨论模型的实用性和可行性\cite{mckean1970nagumo}。

%----------- 参考文献 ----------
\bibliographystyle{unsrt} %规定了参考文献的格式
\begin{center}
\bibliography{reference} %调出LaTeX生成参考文献列表
\end{center}
\textcolor{red}{(所有引用他人或公开资料(包括网上资料)的成果必须按照科技论文的规范列出参考文献,并在正文引用处予以标注。}

\textcolor{red}{常见的三种参考文献的表达方式(标准不唯一):
书籍的表述方式为: [编号] 作者,书名,出版地:出版社,出版年月。
期刊杂志论文的表述方式为: [编号] 作者,论文名,杂志名,卷期号:起止页码,出版年。
网上资源(例如数据库、政府报告)的表述方式为: [编号] 作者,资源标题,网址,访问时间。)}
%----------- 附录 ----------
\newpage
\section{附录}

\begin{table}[htbp]
    \centering
    \begin{tabular}{|p{14.0cm}|}
 %指定单元格宽度, 并且水平居中。
    \hline
    \textbf{附录1} \\ %换行 
    \hline
    介绍:支撑材料的文件列表  \\ 
    \\
    \\
    \\
    \hline
    \end{tabular}
\end{table}

\begin{table}[htbp]
    \centering
    \begin{tabular}{|p{14.0cm}|}
 %指定单元格宽度, 并且水平居中。
    \hline
    \textbf{附录2} \\ %换行 
    \hline
    介绍:该代码是某某语言编写的,作用是什么   \\ 
    \\
    \\
    \\
    \hline
    \end{tabular}
\end{table}

除了支撑材料的文件列表和源程序代码外,附录中还可以包括下面内容:
\begin{itemize}
\item 某一问题的详细证明或求解过程;
\item 自己在网上找到的数据;
\item 比较大的流程图;
\item 较繁杂的图表或计算结果
\end{itemize}

\end{document}